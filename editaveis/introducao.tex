\chapter[Introdução]{Introdução}\label{cap1}

\section{Problema Geral}
A piscicultura é a atividade caracterizada pela criação de peixes em cativeiros, é um ramo da aquicultura, a qual pode ser encontrada na forma de criação de peixes marinhos ou de água doce. Há forte tendência de aumento dessa atividade, devido ao crescimento acelerado da demanda de consumo e a proibição da caça de algumas espécies nativas.

O processo de criação se dá em tanques construídos artificialmente ou em tanques de rede localizados no meio de lagos, rios ou no oceano. Os peixes são alimentados com base em uma dieta de ração ou presa, específica para algumas espécies, diariamente.

A mão de obra disponível e utilizada para atividades operacionais relacionadas à criação é de nível fundamental a médio e os métodos utilizados no manejo em sua maioria ainda são desprovidos de suporte tecnológico e maquinários.

\section{Problemas Específicos}

Existem fatores importantes na qualidade da criação e no desenvolvimento dos animais dentro dos criatórios. A quantidade de ração necessária por indivíduo no tanque varia de acordo com o desenvolvimento corporal da população em cada tanque, além de outros fatores, os quais também influenciam,  como a qualidade da oxigenação da água, ph da água e temperatura.

O problema especificado pelo cliente, o qual possui criatório na represa da usina hidroelétrica em Tocantins, aborda basicamente a solução do problema da qualidade da alimentação realizada manualmente no meio da represa. Atualmente a ração é transportada através de barco e lançada nos tanques, através de um recipiente de quantidade determinada, embora o controle seja relativamente impreciso.

Segundo o cliente, o problema já foi solucionado em alguns criatórios, porém utilizaram dispositivos temporizadores que incorreram em imprecisão e falha no controle da periodicidade da alimentação; foi mencionado a possibilidade de haver falha do equipamento incorrendo na falta de alimentação dos animais por dias.

Há também a necessidade  de supervisão dos dados por meio de um sistema que cumpra as mesmas funcionalidades propostas pelo SCADA (Supervisory Control and Data Acquisition), pois com este sistema se torna possível o controle da alimentação em função Biomassa e Taxa de Conversão Alimentar (TCA) de acordo com as fases de desenvolvimento corporal de cada raça de peixe.

As características prioritárias do sistema são:
\begin{itemize}
  \item Robustez e Resistência de Operação na água.
    \begin{itemize}
      \item Se necessário IP68 em determinados subsistemas.
    \end{itemize}
  \item Operação Simplificada.
    \begin{itemize}
      \item Considerando o grau de escolaridade médio da mão de obra rural.
    \end{itemize}
  \item Confiabilidade.
    \begin{itemize}
      \item Redundância de determinados subsistemas caso necessário, pois determinadas falhas podem causar mortalidade.
    \end{itemize}
  \item SCADA (Supervisory Control and Data Acquisition).
   \begin{itemize}
     \item São sistemas que utilizam software para monitorar a aquisição de dados e controlar dispositivos que possam atuar para controlar as variáveis de interesse do sistema.
   \end{itemize}
\end{itemize}

No caso deste projeto, as variáveis como tempo e quantidade de ração fundamentadas nas características de desenvolvimento corpóreo da raça são as mais importantes variáveis de interesse do sistema.

Características Desejáveis do Sistema:
\begin{itemize}
  \item Monitoramento do ph da água
  \item Monitoramento da temperatura da água
  \item Monitoramento do nível de oxigenação da água
\end{itemize}

O monitoramento do ph, temperatura e nível de oxigenação da água a longo prazo permitirá ao produtor observar o desenvolvimento dos animais de acordo com os parâmetros indicados para a criação das respectivas raças; além de conter dados das características de acidez, temperatura e oxigenação da água ao longo do tempo e poder utilizar esses dados para investigar eventuais discrepâncias no desenvolvimento, causas de mortalidade, índices de fecundação.

\section{Pesquisa de Mercado}

\textbf{FALTA FAZER}

Produtos presentes no mercado, acessibilidade qualidade, referências nacionais e internacionais

\section{Objetivo Geral}
A proposta do grupo é pesquisar e projetar o sistema de alimentação automatizado, o qual atenda os requisitos mínimos impostos pelo cliente e apresentar a melhor escolha em relação ao custo benefício a médio e longo prazo.

\section{Objetivo Específico}
\textbf{FALTA FAZER}
\section{Equipe e Responsabilidades}
\textbf{FALTA FAZER}
\section{Política e Comunicação da Equipe}
\textbf{FALTA FAZER}
\section{Organização do Trabalho}
\textbf{FALTA FAZER}
