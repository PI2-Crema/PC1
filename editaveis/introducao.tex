\chapter[Introdução]{Introdução}\label{cap1}

\section{Problema Geral}
A piscicultura é a atividade caracterizada pela criação de peixes em cativeiros, é um ramo da aquicultura, a qual pode ser encontrada na forma de criação de peixes marinhos ou de água doce. Há forte tendência de aumento dessa atividade, devido ao crescimento acelerado da demanda de consumo e a proibição da caça de algumas espécies nativas.

O processo de criação se dá em tanques construídos artificialmente ou em tanques de rede localizados no meio de lagos, rios ou no oceano. Os peixes são alimentados com base em uma dieta de ração ou presa, específica para algumas espécies, diariamente.

A mão de obra disponível e utilizada para atividades operacionais relacionadas à criação é de nível fundamental a médio e os métodos utilizados no manejo em sua maioria ainda são desprovidos de suporte tecnológico e maquinários.

\section{Problemas Específicos}

Existem fatores importantes na qualidade da criação e no desenvolvimento dos animais dentro dos criatórios. A quantidade de ração necessária por indivíduo no tanque varia de acordo com o desenvolvimento corporal da população em cada tanque, além de outros fatores, os quais também influenciam,  como a qualidade da oxigenação da água, ph da água e temperatura.

O problema especificado pelo cliente, o qual possui criatório na represa da usina hidroelétrica em Tocantins, aborda basicamente a solução do problema da qualidade da alimentação realizada manualmente no meio da represa. Atualmente a ração é transportada através de barco e lançada nos tanques, através de um recipiente de quantidade determinada, embora o controle seja relativamente impreciso.

Segundo o cliente, o problema já foi solucionado em alguns criatórios, porém utilizaram dispositivos temporizadores que incorreram em imprecisão e falha no controle da periodicidade da alimentação; foi mencionado a possibilidade de haver falha do equipamento incorrendo na falta de alimentação dos animais por dias.

Há também a necessidade  de supervisão dos dados por meio de um sistema que cumpra as mesmas funcionalidades propostas pelo SCADA (Supervisory Control and Data Acquisition), pois com este sistema se torna possível o controle da alimentação em função Biomassa e Taxa de Conversão Alimentar (TCA) de acordo com as fases de desenvolvimento corporal de cada raça de peixe.

As características prioritárias do sistema são:
\begin{itemize}
  \item Robustez e Resistência de Operação na água.
    \begin{itemize}
      \item Se necessário IP68 em determinados subsistemas.
    \end{itemize}
  \item Operação Simplificada.
    \begin{itemize}
      \item Considerando o grau de escolaridade médio da mão de obra rural.
    \end{itemize}
  \item Confiabilidade.
    \begin{itemize}
      \item Redundância de determinados subsistemas caso necessário, pois determinadas falhas podem causar mortalidade.
    \end{itemize}
  \item SCADA (Supervisory Control and Data Acquisition).
   \begin{itemize}
     \item São sistemas que utilizam software para monitorar a aquisição de dados e controlar dispositivos que possam atuar para controlar as variáveis de interesse do sistema.
   \end{itemize}
\end{itemize}

No caso deste projeto, as variáveis como tempo e quantidade de ração fundamentadas nas características de desenvolvimento corpóreo da raça são as mais importantes variáveis de interesse do sistema.

Características Desejáveis do Sistema:
\begin{itemize}
  \item Monitoramento do ph da água
  \item Monitoramento da temperatura da água
  \item Monitoramento do nível de oxigenação da água
\end{itemize}

O monitoramento do ph, temperatura e nível de oxigenação da água a longo prazo permitirá ao produtor observar o desenvolvimento dos animais de acordo com os parâmetros indicados para a criação das respectivas raças; além de conter dados das características de acidez, temperatura e oxigenação da água ao longo do tempo e poder utilizar esses dados para investigar eventuais discrepâncias no desenvolvimento, causas de mortalidade, índices de fecundação.

\section{Pesquisa de Mercado}

No mercado existe uma quantidade considerável de alimentadores automáticos porém poucas são as soluções que apresentam a característica flutuante. A grande maioria são para estruturas para apoio terrestre e com uma baixa capacidade para a quantidade de ração. Uma solução flutuante para tanques de rede encontrada é a criada pela empresa Marcsystem. Se trata de um alimentador para peixes com capacidade nominal para 60 kg de ração e tem sua estrutura em plástico numa dosagem de até 1kg em 7s\cite{Marcsystem}. Porém o custo do produto não está especificado e o grupo não teve resposta do fabricante até o momento. Comparado com outras opções da própria empresa temos um tratador automático para cães com uma capacidade de 25kg, tendo o custo de R\$ 890,00 e não conta com  uma estrutura flutuante. Vale ressaltar que uma estrutura flutuante deve contar com materiais mais resistentes e leve, o que encarece o projeto.
Essa foi a única solução que se encaixa na especificidade flutuante e de alimentação de peixes do projeto. E ainda assim não tem a autonomia de 300 kg de ração especificada pelo cliente. Podemos então verificar que o projeto ocupa uma lacuna no mercado nacional, tratadores com grande autonomia para alimentação de peixes em lagos.


Produtos presentes no mercado, acessibilidade qualidade, referências nacionais e internacionais

\section{Objetivo Geral}
A proposta do grupo é pesquisar e projetar o sistema de alimentação automatizado, o qual atenda os requisitos mínimos impostos pelo cliente e apresentar a melhor escolha em relação ao custo benefício a médio e longo prazo.

\section{Objetivo Específico}

\begin{itemize}
  \item Despejar quantia específica de ração em períodos pré-determinados;
  \item Medir parâmetros da água em que os tanques se encontram relevantes à piscicultura;
  \item Conectar os tanques a uma central através de uma rede sem fio;
  \item Disponibilizar os dados das medições através de uma aplicação web;
  \item Carregar os baterias com o uso de painéis fotovoltaicos;
\end{itemize}

\section{Equipe e Responsabilidades}

A equipe resposável pelo desenvolvimento do projeto está distribuida de acordo com a Tabela \ref{equipe}. Sendo formada por 12 integrantes (estudantes das cinco engenharias da UNB Gama).

\begin{table}[H]
\centering
\caption{Equipe do projeto Crema}
\label{equipe}
\begin{tabular}{|l|l|l|}
\hline
Nome                           & Matrícula  & Curso              \\ \hline
Barbara Helen da Silva         & 13/0103241 & Eng. de Energia    \\ \hline
Emilly Caroline Costa Silva    & 13/0008524 & Eng. de Energia    \\ \hline
Fabiana Campos Ribeiro         & 10/0100040 & Eng. de Energia    \\ \hline
Daniel Borges Pinheiro         & 12/0114283 & Eng. de Eletrônica \\ \hline
Filipe Batista Ribeiro Costa   & 12/0117673 & Eng. de Eletrônica \\ \hline
Teles Mauricio Presa Raulinho  & 12/0042274 & Eng. de Eletrônica \\ \hline
Eliseu Egewarth                & 12/0029693 & Eng. de Software   \\ \hline
Gabriel de Araújo Silva        & 12/0118220 & Eng. de Software   \\ \hline
Luis Filipe Resende Vilela     & 12/0036754 & Eng. de Software   \\ \hline
Rafael Alarcão Uchôa Tenorio   & 13/0035912 & Eng. Automotiva    \\ \hline
Tulio Costa Oliveira           & 12/0062208 & Eng. Automotiva    \\ \hline
Luis Fernando Marzola de Cunha & 12/0017156 & Eng. Aeroespacial  \\ \hline
\end{tabular}
\end{table}

É importante, como equipe, que todos os integrantes estejam a par do que está acontencendo no projeto e quais são os próximos passos e em caso de dúvida à quem recorrer. Dessa forma foi definido que a metodologia SCRUM será utilizada e que em cada sprint o papel de SCRUM MASTER passará para outro estudante. Dessa forma não há uma sobre cargar em um único integrante.

A equipe também definiu um resposável por cada área de atuação do projeto. Este, por sua vez, é responsável por fazer com que a equipe resposável pelo módulo trabalhe em conjunto. A distribuição de de responsábilidades está de acordo com a tabela \ref{atuacao}.

% Please add the following required packages to your document preamble:
\begin{table}[H]
\centering
\caption{Equipe - Áreas de atuação}
\label{atuacao}
\begin{tabular}{|c|l|}
\hline
\multirow{3}{*}{Eletrônica}                                            & \multicolumn{1}{c|}{Sensoriamento}                                                    \\ \cline{2-2}
                                                                       & Atuadores                                                                             \\ \cline{2-2}
                                                                       & \multicolumn{1}{c|}{\multirow{2}{*}{Comunicação}}                                     \\ \cline{1-1}
\multirow{2}{*}{Software}                                              & \multicolumn{1}{c|}{}                                                                 \\ \cline{2-2}
                                                                       & \begin{tabular}[c]{@{}l@{}}Interface de\\ Controle e análise \\ de dados\end{tabular} \\ \hline
Energia                                                                & Alimentação                                                                         \\ \hline
\begin{tabular}[c]{@{}c@{}}Automotiva \\ e\\ Aeroespacial\end{tabular} & Estrutura                                                                             \\ \hline
\end{tabular}
\end{table}

\section{Política e Comunicação da Equipe}

Para facilitar a organização e a comunicação interna do grupo, foram selecionas algumas ferramentas gratuitas. Para a comunicação foi selecionado o aplicativo de mensagem instantanea Whatsapp.
Para o desenvolvimento de documentação e a distribuição de documentos e ou recursos foi selecionado a ferramente Google Drive que permite o compartilhamento online à um único \textit{hub} de documentos.

Uma vez que o metodologia de gerênciamento SCRUM foi seleciona a ferramenta trello será utilizada como o KANBAN da equipe.

A técnica de \textit{daily meeting} serão realizadas de forma virtual através da ferramenta de chat Whatsapp. Todos deram response ao final do dia as seguintes perguntas: \textit{o que foi feito hoje?, o que será feito amanhã? e o que está me bloqueando?}.

\section{Organização do Trabalho}

Este documento está organizado para apresentar, primeiramente, uma visão geral dos requisitos e as limitações defindas. Em seguida uma visão geral da solução proposta, passando, de formar superficial, os módulos da solução. Após isso serão detalhados cada módulo, o que são, quais materiais e como serão utilizados. Os módulos da solução são: módulo de energia(responsável por fornecer energia elétrica), módulo de controle e comunicação (resposável por controlar o funcionamento do produto e disponibilizar uma interface de comunicação com o usuário), módulos de estrutura (responsável por definir a forma e materiais utilizados pela solução).
